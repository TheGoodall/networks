\documentclass{article}


\usepackage[margin=0.5in]{geometry}
\usepackage{verbatim}

\usepackage{parskip}
\usepackage{tikz,pgfplots}

\renewcommand{\thesubsection}{\arabic{subsection}}

\title{Network Science Topic 1}
\date{}


\begin{document}
\maketitle

\section{}
\subsection{}
\subsubsection{Largest connected component size}
The number of vertices in the largest connected component of
\verb"citation_graph" 
is \input{G_Size.data}

\subsubsection{Out degree distribution of G}

\begin{tikzpicture}
\begin{semilogyaxis}[scatter/classes={a={mark=o,draw=black}}]
\addplot[scatter,only marks,scatter src=explicit symbolic]
table[x=degree, y=count, col sep=comma] {out_distro.data};
\end{semilogyaxis}
\end{tikzpicture}

\subsubsection{In degree distribution of G}

\begin{tikzpicture}
\begin{semilogyaxis}[scatter/classes={a={mark=o,draw=black}}]
\addplot[scatter,only marks,scatter src=explicit symbolic]
table[x=degree, y=count, col sep=comma] {in_distro.data};
\end{semilogyaxis}
\end{tikzpicture}

\subsection{}

\subsubsection{Out degree distribution of PA graph}

\begin{tikzpicture}
\begin{semilogyaxis}[scatter/classes={a={mark=o,draw=black}}]
\addplot[scatter,only marks,scatter src=explicit symbolic]
table[x=degree, y=count, col sep=comma] {PA_out_distro.data};
\end{semilogyaxis}
\end{tikzpicture}

The out-degree distribution used is an exponential distribution. It is fairly similar to to out-degree distribution of G, but is significantly more noisy.
Most papers will not cite many other papers, but some will cite many others, thus an exponential distribution is relevent.

\subsubsection{In degree distribution of PA graph}

\begin{tikzpicture}
\begin{semilogyaxis}[scatter/classes={a={mark=o,draw=black}}]
\addplot[scatter,only marks,scatter src=explicit symbolic]
table[x=degree, y=count, col sep=comma] {PA_in_distro.data};
\end{semilogyaxis}
\end{tikzpicture}

The in-degree distribution used is proportional to the in-degree of a node, e.g. each node that is added will prefer to connect to nodes that have a high number of nodes already connected.
e.g. papers released will often cite papers that have already been cited a large amount, and not many papers will chose to cite less cited papers.


\end{document}
